%%%%%%%%%%%%%%%%%%%%%%%%%%%%%%%%%%%%%%%%%%%%%%%%%%
%%	by: Ehsan Emamjomeh-Zadeh
%%	http://www-scf.usc.edu/~emamjome/
%%
%%	Contact information:
%%		emamjome[at]usc[dot]edu
%%		ehsan7069[at]gmail[dot]com
%%
%%	current version: 3.2
%%	last update: Nov. 30, 2015
%%%%%%%%%%%%%%%%%%%%%%%%%%%%%%%%%%%%%%%%%%%%%%%%%%


%%%%%%%%%%%%%%%%%%%%%%%%%%%%%%%%%%%%%%%%%%%%%%%%%%
%%	by: Ehsan Emamjomeh-Zadeh
%%	http://www-scf.usc.edu/~emamjome/
%%
%%	Contact information:
%%		emamjome[at]usc[dot]edu
%%		ehsan7069[at]gmail[dot]com
%%
%%	current version: 3.2
%%	last update: Nov. 30, 2015
%%%%%%%%%%%%%%%%%%%%%%%%%%%%%%%%%%%%%%%%%%%%%%%%%%


\documentclass[11pt]{article}

\usepackage{fullpage}
\usepackage{nopageno} % removing page numbers
\usepackage{xspace}
\usepackage{xcolor}
\usepackage{ifthen}
  
\newcommand{\myDef}[2]{%
\expandafter\def\csname #1\endcsname{#2\xspace}}

\newboolean{inHeaderShowNote}
\newboolean{inHeaderShowFullName}
\newboolean{inHeaderShowDate}
\newboolean{inSignitureShowFullName}
\newboolean{inSignitureShowDate}

\newlength{\PushHeaderUp}
\newlength{\SpaceBeforeNote}
\newlength{\SpaceBeforeFullName}
\newlength{\SpaceBeforeDate}
\newlength{\SpaceAfterHeader}
\newlength{\SpaceBeforeSigniture}


\def\mySigniture{%
\vspace*{\SpaceBeforeSigniture}%
\ifthenelse{\equal{\SigniturePosition}{left}}%
{\flushleft}{\flushright}{%
\begin{minipage}{\SignitureWidth\textwidth}
\ifthenelse{\boolean{inSignitureShowFullName}}%
{\par{\FullName}}{}%
\ifthenelse{\boolean{inSignitureShowDate}}%
{\par{\Date}}{}
\end{minipage}}}



%\usepackage{../base/ehsan7069}

%%%%%%%%%%%%%%%%%%%%%%%%%%%%%%%%%%%%%%%%%%%%%%%%%%
%%	by: Ehsan Emamjomeh-Zadeh
%%	http://www-scf.usc.edu/~emamjome/
%%
%%	Contact information:
%%		emamjome[at]usc[dot]edu
%%		ehsan7069[at]gmail[dot]com
%%
%%	current version: 3.2
%%	last update: Nov. 30, 2015
%%%%%%%%%%%%%%%%%%%%%%%%%%%%%%%%%%%%%%%%%%%%%%%%%%


%%%%%%%%%%%%%%%%%%%%%%%%%%%%%%%%%%%%%%%%%%%%%%%%%%
%%	Feel free to leave some info empty.
%%%%%%%%%%%%%%%%%%%%%%%%%%%%%%%%%%%%%%%%%%%%%%%%%%

%% full name, to be shown on top:
\def\FullName{Ehsan Emamjomeh-Zadeh}

%% contact info, for the left side:
\def\DepartmentName{Department of Computer Science}
\def\UniversityName{University of Southern California}
\def\UniversityAddress{Los Angeles, CA 90089, United States}
\def\OfficePhoneNumber{}

%% contact info, for the right side:
\def\MainEmailAddress{emamjome@usc.edu}
\def\SecondEmailAddress{ehsan7069@gmail.com}
\def\CellPhone{}
\def\Homepage{http://www-scf.usc.edu/~emamjome/}


%%%%%%%%%%%%%%%%%%%%%%%%%%%%%%%%%%%%%%%%%%%%%%%%%%
%%	You wouldn't probably need
%%	to touch the rest of file.
%%%%%%%%%%%%%%%%%%%%%%%%%%%%%%%%%%%%%%%%%%%%%%%%%%

\author{\FullName}

\contactInfo{%
\ifthenelse{\equal{\DepartmentName}{}}{}%
{\DepartmentName\\}
\ifthenelse{\equal{\UniversityName}{}}{}%
{\UniversityName\\}
\ifthenelse{\equal{\UniversityAddress}{}}{}%
{\UniversityAddress\\}
\ifthenelse{\equal{\OfficePhoneNumber}{}}{}%
{\OfficePhoneNumber\\}
}%
{%
\ifthenelse{\equal{\MainEmailAddress}{}}{}%
{\href{mailto:\MainEmailAddress}{\textrm{\MainEmailAddress}}\\}
\ifthenelse{\equal{\SecondEmailAddress}{}}{}%
{\href{mailto:\SecondEmailAddress}{\textrm{\SecondEmailAddress}}\\}
\ifthenelse{\equal{\CellPhone}{}}{}%
{Cell Phone: \CellPhone\\}
\ifthenelse{\equal{\Homepage}{}}{}%
{\href{\Homepage}{\url{\Homepage}}\\}
Last Update: \today\\
}



\begin{document}

%%%%%%%%%%%%%%%%%%%%%%%%%%%%%%%%%%%%%%%%%%%%%%%%%%
%%	by: Ehsan Emamjomeh-Zadeh
%%	http://www-scf.usc.edu/~emamjome/
%%
%%	Contact information:
%%		emamjome[at]usc[dot]edu
%%		ehsan7069[at]gmail[dot]com
%%
%%	current version: 3.2
%%	last update: Nov. 30, 2015
%%%%%%%%%%%%%%%%%%%%%%%%%%%%%%%%%%%%%%%%%%%%%%%%%%


\makeatletter
\def\@maketitle{%
	\begin{center}
		\vspace*{-\PushHeaderUp}
		\par %% title
		\begin{Large}
			Statement of Purpose
		\end{Large}
		\par
		%% Note
		\ifthenelse{\boolean{inHeaderShowNote}}%
		{\vspace*{\SpaceBeforeNote}\Note\par}{}
		%% FullName
		\ifthenelse{\boolean{inHeaderShowFullName}}%
		{\vspace*{\SpaceBeforeFullName}%
		\begin{LARGE}\textit{\FullName}\end{LARGE}\par}{}
		%% Date
		\ifthenelse{\boolean{inHeaderShowDate}}%
		{\vspace*{\SpaceBeforeDate}\Date\par}{}
		\vspace*{\SpaceAfterHeader}
	\end{center}}
\maketitle

%% spacings in file:

\ifthenelse{\equal{\IndentParagraph}{}}{}%
{\setlength{\parindent}{\IndentParagraph em}}
\ifthenelse{\equal{\SpaceBetweenParagraphs}{}}{}%
{\setlength{\parskip}{\SpaceBetweenParagraphs ex}}



First of all, take a look at \emph{info.tex}
and fill it out.
Feel free to add your favorite commands to this file,
before you begin to write anything.
You can find a couple of examples there.
For instance, it's probably safer
(and definitely more fun) to use
\textcolor{red}{\textbackslash FirstProf},
\textcolor{red}{\textbackslash SecondProf}, and
\textcolor{red}{\textbackslash ThirdProf},
letting \LaTeX{} to replace them
with actual names.
(Here is what happens:
\FirstProf, \SecondProf, and \ThirdProf.)
You can also find a bunch of boolean variables
in \emph{info.tex}
indicating what appear in the header and what don't.
Finally, there are some length variables
controlling the spacings.
I hope these variables (all provided in \emph{info.tex})
make the template flexible enough
so you wouldn't need to touch other files.
Yet if there are some other things you like to change,
feel free to take a look at
\emph{SoP-style.tex} (for general style)
and \emph{header.tex}
(for the header and some other things).

For those of you who are not familiar with \LaTeX,
the following basic commands may be useful:
\begin{itemize}
\itemsep0em
\item \textbackslash textit for \textit{italic};
\item \textbackslash textbf for \textbf{bold};
\item \textbackslash underline for \underline{underline}.
\end{itemize}

If there is any problem with this template,
don't hesitate to contact me :D Good luck!

%% Position of the signiture (left or right)
%% is determined by \SigniturePosition (in info.tex).
\mySigniture

\end{document}

