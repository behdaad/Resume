%%%%%%%%%%%%%%%%%%%%%%%%%%%%%%%%%%%%%%%%%%%%%%%%%%
%%	by: Ehsan Emamjomeh-Zadeh
%%	http://www-scf.usc.edu/~emamjome/
%%
%%	Contact information:
%%		emamjome[at]usc[dot]edu
%%		ehsan7069[at]gmail[dot]com
%%
%%	current version: 3.2
%%	last update: Nov. 30, 2015
%%%%%%%%%%%%%%%%%%%%%%%%%%%%%%%%%%%%%%%%%%%%%%%%%%


%%%%%%%%%%%%%%%%%%%%%%%%%%%%%%%%%%%%%%%%%%%%%%%%%%
%%	by: Ehsan Emamjomeh-Zadeh
%%	http://www-scf.usc.edu/~emamjome/
%%
%%	Contact information:
%%		emamjome[at]usc[dot]edu
%%		ehsan7069[at]gmail[dot]com
%%
%%	current version: 3.2
%%	last update: Nov. 30, 2015
%%%%%%%%%%%%%%%%%%%%%%%%%%%%%%%%%%%%%%%%%%%%%%%%%%


\documentclass[11pt]{article}

\usepackage{fullpage}
\usepackage{nopageno} % removing page numbers
\usepackage{xspace}
\usepackage{xcolor}
\usepackage{ifthen}
  
\newcommand{\myDef}[2]{%
\expandafter\def\csname #1\endcsname{#2\xspace}}

\newboolean{inHeaderShowNote}
\newboolean{inHeaderShowFullName}
\newboolean{inHeaderShowDate}
\newboolean{inSignitureShowFullName}
\newboolean{inSignitureShowDate}

\newlength{\PushHeaderUp}
\newlength{\SpaceBeforeNote}
\newlength{\SpaceBeforeFullName}
\newlength{\SpaceBeforeDate}
\newlength{\SpaceAfterHeader}
\newlength{\SpaceBeforeSigniture}


\def\mySigniture{%
\vspace*{\SpaceBeforeSigniture}%
\ifthenelse{\equal{\SigniturePosition}{left}}%
{\flushleft}{\flushright}{%
\begin{minipage}{\SignitureWidth\textwidth}
\ifthenelse{\boolean{inSignitureShowFullName}}%
{\par{\FullName}}{}%
\ifthenelse{\boolean{inSignitureShowDate}}%
{\par{\Date}}{}
\end{minipage}}}



%\usepackage{../base/ehsan7069}

%%%%%%%%%%%%%%%%%%%%%%%%%%%%%%%%%%%%%%%%%%%%%%%%%%
%%	by: Ehsan Emamjomeh-Zadeh
%%	http://www-scf.usc.edu/~emamjome/
%%
%%	Contact information:
%%		emamjome[at]usc[dot]edu
%%		ehsan7069[at]gmail[dot]com
%%
%%	current version: 3.2
%%	last update: Nov. 30, 2015
%%%%%%%%%%%%%%%%%%%%%%%%%%%%%%%%%%%%%%%%%%%%%%%%%%


%% personal information:

\myDef{FullName}{Ehsan Emamjomeh-Zadeh}
\myDef{Date}{\today} % usually \today (unless you wanna manually change it)
\myDef{Note}{University of Southern California} % to appear in the header

%% The followings are optional.
%% You can have as many of them as you like.
\myDef{UniversityName}{University of Southern California} % optional
\myDef{UniversityNameShort}{USC} % optional
\myDef{FirstResearchInterest}{Theoretical Computer Science} % optional
\myDef{FirstProf}{Professor X}	% optional
\myDef{SecondProf}{Professor Y}	% optional
\myDef{ThirdProf}{Professor Z}	% optional


%%%%%%%%%%%%%%%%%%%%%%%%%%%%%%%%%%%%%%%%%%%%%%%%%%
%% booleans to modify the header and your signiture:

\setboolean{inHeaderShowNote}{true}
\setboolean{inHeaderShowFullName}{true}
\setboolean{inHeaderShowDate}{false}
\setboolean{inSignitureShowFullName}{true}
\setboolean{inSignitureShowDate}{true}


%%%%%%%%%%%%%%%%%%%%%%%%%%%%%%%%%%%%%%%%%%%%%%%%%%
%% spacing:

\setlength{\PushHeaderUp}{6ex} % possibly negative
%% Each of the following lengths is effective
%% only if the corresponding boolean is true.
\setlength{\SpaceBeforeNote}{2ex}
\setlength{\SpaceBeforeFullName}{4ex}
\setlength{\SpaceBeforeDate}{2ex}
\setlength{\SpaceAfterHeader}{2ex}

\setlength{\SpaceBeforeSigniture}{2ex}
\def\SignitureWidth{0.40} % in range (0, 1)

%% where to insert the signiture? left or right?
\def\SigniturePosition{left} % either left or right

%% You can leave each of the followings emply
%% if you're happy with its default value.
\def\IndentParagraph{2.0} % indent the first line of each paragraph
\def\SpaceBetweenParagraphs{1.2} % space between consecutive paragraphs
\renewcommand{\baselinestretch}{1.1} % space between lines within a paragraph



\begin{document}

%%%%%%%%%%%%%%%%%%%%%%%%%%%%%%%%%%%%%%%%%%%%%%%%%%
%%	by: Ehsan Emamjomeh-Zadeh
%%	http://www-scf.usc.edu/~emamjome/
%%
%%	Contact information:
%%		emamjome[at]usc[dot]edu
%%		ehsan7069[at]gmail[dot]com
%%
%%	current version: 3.2
%%	last update: Nov. 30, 2015
%%%%%%%%%%%%%%%%%%%%%%%%%%%%%%%%%%%%%%%%%%%%%%%%%%


\makeatletter
\def\@maketitle{%
	\begin{center}
		\vspace*{-\PushHeaderUp}
		\par %% title
		\begin{Large}
			Statement of Purpose
		\end{Large}
		\par
		%% Note
		\ifthenelse{\boolean{inHeaderShowNote}}%
		{\vspace*{\SpaceBeforeNote}\Note\par}{}
		%% FullName
		\ifthenelse{\boolean{inHeaderShowFullName}}%
		{\vspace*{\SpaceBeforeFullName}%
		\begin{LARGE}\textit{\FullName}\end{LARGE}\par}{}
		%% Date
		\ifthenelse{\boolean{inHeaderShowDate}}%
		{\vspace*{\SpaceBeforeDate}\Date\par}{}
		\vspace*{\SpaceAfterHeader}
	\end{center}}
\maketitle

%% spacings in file:

\ifthenelse{\equal{\IndentParagraph}{}}{}%
{\setlength{\parindent}{\IndentParagraph em}}
\ifthenelse{\equal{\SpaceBetweenParagraphs}{}}{}%
{\setlength{\parskip}{\SpaceBetweenParagraphs ex}}



First of all, take a look at \emph{info.tex}
and fill it out.
Feel free to add your favorite commands to this file,
before you begin to write anything.
You can find a couple of examples there.
For instance, it's probably safer
(and definitely more fun) to use
\textcolor{red}{\textbackslash FirstProf},
\textcolor{red}{\textbackslash SecondProf}, and
\textcolor{red}{\textbackslash ThirdProf},
letting \LaTeX{} to replace them
with actual names.
(Here is what happens:
\FirstProf, \SecondProf, and \ThirdProf.)
You can also find a bunch of boolean variables
in \emph{info.tex}
indicating what appear in the header and what don't.
Finally, there are some length variables
controlling the spacings.
I hope these variables (all provided in \emph{info.tex})
make the template flexible enough
so you wouldn't need to touch other files.
Yet if there are some other things you like to change,
feel free to take a look at
\emph{SoP-style.tex} (for general style)
and \emph{header.tex}
(for the header and some other things).

For those of you who are not familiar with \LaTeX,
the following basic commands may be useful:
\begin{itemize}
\itemsep0em
\item \textbackslash textit for \textit{italic};
\item \textbackslash textbf for \textbf{bold};
\item \textbackslash underline for \underline{underline}.
\end{itemize}

If there is any problem with this template,
don't hesitate to contact me :D Good luck!

%% Position of the signiture (left or right)
%% is determined by \SigniturePosition (in info.tex).
\mySigniture

\end{document}

